%\newtheorem{example}{Example}

\newcommand{\expect}{\mathbb{E}}
\newcommand{\variance}{\mathrm{Var}}
\newcommand{\covariance}{\mathrm{Cov}}
\newcommand{\prob}{\mathrm{Pr}}
\newcommand{\zeroVec}{\mathbf{0}}
\newcommand{\zeroMat}{\mathbf{0}}
\newcommand{\onesVec}{\mathbf{1}}
\newcommand{\deriv}{\mathrm{d}}
\newcommand{\transpose}{\top}
\newcommand{\costDeriv}[1]{\overline{#1}}
\newcommand{\lossDeriv}{\costDeriv}
\newcommand{\normal}{\mathcal{N}}
\newcommand{\data}{\mathcal{D}}
\newcommand{\kldiv}{\mathrm{D}_{\mathrm{KL}}}
\DeclareMathOperator{\diag}{diag}


\newcommand{\dataIdx}{i}
\newcommand{\featIdx}{j}
\newcommand{\dimIdx}{\featIdx}
\newcommand{\paramIdx}{\dimIdx}
\newcommand{\hidIdx}{i}
\newcommand{\classIdx}{k}
\newcommand{\outputIdx}{k}
\newcommand{\classIdxTwo}{\ell}
\newcommand{\featIdxTwo}{j^\prime}
\newcommand{\nfeat}{D}
\newcommand{\ndim}{\nfeat}
\newcommand{\ndata}{N}
\newcommand{\numClasses}{K}
\newcommand{\nout}{\numClasses}
\newcommand{\layerIdx}{\ell}
\newcommand{\numLayers}{L}
\newcommand{\nhid}{M}
\newcommand{\timeIdx}{t}
\newcommand{\ntime}{T}
\newcommand{\contextLen}{K}
\newcommand{\iterIdx}{k}


\newcommand{\inputIJ}[2]{x^{(#1)}_{#2}}
\newcommand{\inputI}[1]{{\bf x}^{(#1)}}
\newcommand{\inputJ}[1]{x_{#1}}
\newcommand{\inputVec}{{\bf x}}
\newcommand{\inputVecI}[1]{\inputVec^{(#1)}}
\newcommand{\inputVecT}[1]{\inputVec^{(#1)}}
\newcommand{\inputUni}{x}
\newcommand{\inputUniI}[1]{x^{(#1)}}
\newcommand{\inputUniT}[1]{x^{(#1)}}
\newcommand{\inputMatrix}{\mathbf{X}}
\newcommand{\inputMatrixT}[1]{\inputMatrix^{(#1)}}
\newcommand{\targetI}[1]{t^{(#1)}}
\newcommand{\target}{t}
\newcommand{\targetK}[1]{\target_{#1}}
\newcommand{\targets}{\mathbf{t}}
\newcommand{\prediction}{y}
\newcommand{\predictionI}[1]{y^{(#1)}}
\newcommand{\predictionK}[1]{y_{#1}}
\newcommand{\predictionT}[1]{y^{(#1)}}
\newcommand{\predictions}{\mathbf{y}}
\newcommand{\predictionMatrix}{\mathbf{Y}}
\newcommand{\predictionMatrixT}[1]{\predictionMatrix^{(#1)}}
\newcommand{\intermediate}{z}
\newcommand{\intermediateI}[1]{\intermediate^{(#1)}}
\newcommand{\intermediateT}[1]{\intermediate^{(#1)}}
\newcommand{\intermediateK}[1]{\intermediate_{#1}}
\newcommand{\intermediates}{\mathbf{z}}
\newcommand{\intermediateMatrix}{\mathbf{Z}}
\newcommand{\intermediateMatrixT}[1]{\intermediateMatrix^{(#1)}}
\newcommand{\outIntermediate}{r}
\newcommand{\outIntermediateT}[1]{r^{(#1)}}
\newcommand{\outIntermediateK}[1]{\outIntermediate_{#1}}
\newcommand{\outIntermediates}{\mathbf{r}}
\newcommand{\outIntermediateMat}{\mathbf{R}}
\newcommand{\outIntermediateMatrix}{\mathbf{R}}
\newcommand{\outIntermediateMatrixT}[1]{\outIntermediateMatrix^{(#1)}}
\newcommand{\hiddenI}[1]{h_{#1}}
\newcommand{\hiddenT}[1]{h^{(#1)}}
\newcommand{\hiddenIT}[2]{h_{#1}^{(#2)}}
\newcommand{\hiddenLI}[2]{h_{#2}^{(#1)}}
\newcommand{\hiddens}{\mathbf{h}}
\newcommand{\hiddensL}[1]{\hiddens^{(#1)}}
\newcommand{\hiddensT}[1]{\hiddens^{(#1)}}
\newcommand{\hiddenMatrix}{\mathbf{H}}
\newcommand{\hiddenMat}{\hiddenMatrix}
\newcommand{\hiddenMatrixT}[1]{\hiddenMatrix^{(#1)}}
\newcommand{\hiddenMatL}[1]{\hiddenMat^{(#1)}}
\newcommand{\weights}{{\bf w}}
\newcommand{\weightsL}[1]{\weights^{(#1)}}
\newcommand{\weightJ}[1]{w_{#1}}
\newcommand{\weightLIJ}[3]{w^{(#1)}_{#2 #3}}
\newcommand{\weightLKI}[3]{w^{(#1)}_{#2 #3}}
\newcommand{\weightLJ}[2]{w^{(#1)}_{#2}}
\newcommand{\weightKJ}[2]{w_{#1 #2}}
\newcommand{\weightIJ}{\weightKJ}
\newcommand{\weightUni}{w}
\newcommand{\weightMat}{\mathbf{W}}
\newcommand{\weightMatL}[1]{\weightMat^{(#1)}}
\newcommand{\bias}{b}
\newcommand{\biasLI}[2]{\bias^{(#1)}_{#2}}
\newcommand{\biasLK}{\biasLI}
\newcommand{\biasL}[1]{\bias^{(#1)}}
\newcommand{\biasK}[1]{\bias_{#1}}
\newcommand{\biasJ}[1]{\bias_{#1}}
\newcommand{\biases}{\mathbf{b}}
\newcommand{\biasesL}[1]{\biases^{(#1)}}
\newcommand{\threshold}{r}
\newcommand{\featureJ}[1]{\phi_{#1}}
\newcommand{\featureVec}{{\boldsymbol \phi}}
\newcommand{\loss}{\mathcal{L}}
\newcommand{\lossI}[1]{\mathcal{L}^{(#1)}}
\newcommand{\zeroOneLoss}{\loss_{\rm 0-1}}
\newcommand{\squaredErrorLoss}{\loss_{\rm SE}}
\newcommand{\crossEntropyLoss}{\loss_{\rm CE}}
\newcommand{\logisticCrossEntropyLoss}{\loss_{\rm LCE}}
\newcommand{\softmaxCrossEntropyLoss}{\loss_{\rm SCE}}
\newcommand{\hingeLoss}{\loss_{\rm H}}
\newcommand{\cost}{\mathcal{J}}
\newcommand{\regularizer}{\mathcal{R}}
\newcommand{\lrate}{\alpha}
\newcommand{\learningRate}{\lrate}
\newcommand{\featureMap}{{\boldsymbol \psi}}
\newcommand{\featureMapJ}[1]{\psi_{#1}}
\newcommand{\sigmoid}{\sigma}
\newcommand{\logistic}{\sigmoid}
\newcommand{\activationFunction}{\phi}
\newcommand{\activationFunctionL}[1]{\activationFunction^{(#1)}}
\newcommand{\activationFunctionTwo}{\psi}
\newcommand{\parityFunction}{f_{\rm par}}
\newcommand{\function}{f}
\newcommand{\functionL}[1]{\function^{(#1)}}
\newcommand{\indicatorOf}[1]{\mathbbm{1}_{#1}}
\newcommand{\softmax}{\mathrm{softmax}}
\newcommand{\weightCost}{\lambda}
\newcommand{\genCost}{\mathcal{C}}
\newcommand{\momentumVec}{\mathbf{p}}
\newcommand{\momentumJ}[1]{p_{#1}}
\newcommand{\momentumParam}{\mu}
\newcommand{\genParams}{{\boldsymbol \theta}}
\newcommand{\genParamJ}[1]{\theta_{#1}}
\newcommand{\pData}{p_{\mathcal{D}}}
\newcommand{\bestPrediction}{\prediction_\star}
\newcommand{\hessian}{\mathbf{H}}
\newcommand{\jacobian}{\mathbf{J}}

\newcommand{\obs}{\mathbf{x}}
\newcommand{\obsJ}[1]{x_{#1}}
\newcommand{\obsI}[1]{\obs^{(#1)}}
\newcommand{\pfn}{\mathcal{Z}}
\newcommand{\happiness}{H}
\newcommand{\latents}{\mathbf{z}}

\newcommand{\state}{\mathbf{s}}
\newcommand{\stateT}[1]{\state_{#1}}
\newcommand{\act}{\mathbf{a}}
\newcommand{\actT}[1]{\act_{#1}}
\newcommand{\reward}{r}
\newcommand{\policy}{\pi}
\newcommand{\policyParams}{\boldsymbol{\theta}}
\newcommand{\policyTh}{{\policy_{\policyParams}}}
\newcommand{\MDP}{\mathcal{M}}
\newcommand{\rollout}{\tau}
\newcommand{\expectedReturn}{R}

\newcommand{\discReturn}{G}
\newcommand{\discFactor}{\gamma}
\newcommand{\valueFunc}{V}
\newcommand{\valueFuncPi}{\valueFunc^{\policy}}
\newcommand{\valueFuncPiTh}{\valueFunc^{\policyTh}}
\newcommand{\qFunc}{Q}
\newcommand{\qFuncPi}{\qFunc^{\policy}}
\newcommand{\optPolicy}{\policy^*}
\newcommand{\optQ}{\qFunc^*}

\newcommand{\decoderFunc}{G}
\newcommand{\vfe}{\mathcal{F}}
\newcommand{\bmu}{\boldsymbol{\mu}}
\newcommand{\bSigma}{\boldsymbol{\Sigma}}
\newcommand{\bsigma}{\boldsymbol{\sigma}}
\newcommand{\bepsilon}{\boldsymbol{\epsilon}}

\newcommand{\featureMatrix}{{\boldsymbol \Psi}}
\newcommand{\priorMean}{\mathbf{m}}
\newcommand{\priorCov}{\mathbf{S}}
\newcommand{\priorVar}{\eta}
\newcommand{\postMean}{\boldsymbol{\mu}}
\newcommand{\postMeanJ}[1]{\mu_{#1}}
\newcommand{\postCov}{\boldsymbol{\Sigma}}
\newcommand{\postStdJ}[1]{\sigma_{#1}}
\newcommand{\predMean}{\mu_{\rm pred}}
\newcommand{\predVar}{\sigma^2_{\rm pred}}
\newcommand{\predStd}{\sigma_{\rm pred}}


\newcommand{\given}{\,|\,}
\newcommand{\klBars}{\,\|\,}
\newcommand{\TODO}[1]{{\color{blue} {\bf [[#1]]}}}



\newcommand{\naive}{na{\"\i}ve }


%-------------------------------------------------------------------------------------
% Dependencies
%-------------------------------------------------------------------------------------
\usepackage{comment,url,algorithm,algorithmic,graphicx,subcaption,relsize}
\usepackage{amssymb,amsfonts,amsmath,amsthm,amscd,dsfont,mathrsfs,mathtools,nicefrac}
\usepackage{float,psfrag,epsfig,color,xcolor,url,hyperref}
\usepackage{epstopdf,bbm,mathtools,enumitem}


%-------------------------------------------------------------------------------------
% Common differentials with a small space in front of them
%-------------------------------------------------------------------------------------
\newcommand{\dt}{\,\dee t}
\newcommand{\ds}{\,\dee s}
\newcommand{\dx}{\,\dee x}
\newcommand{\dy}{\,\dee y}
\newcommand{\dz}{\,\dee z}
\newcommand{\dv}{\,\dee v}
\newcommand{\dw}{\,\dee w}
\newcommand{\dr}{\,\dee r}
\newcommand{\dB}{\,\dee B} % Brownian motion
\newcommand{\dW}{\,\dee W} % Wiener process
\newcommand{\dmu}{\,\dee \mu}
\newcommand{\dnu}{\,\dee \nu}
\newcommand{\domega}{\,\dee \omega}

%-------------------------------------------------------------------------------------
% Set notation
%-------------------------------------------------------------------------------------
\newcommand{\smiddle}{\mathrel{}|\mathrel{}} % Well-spaced \middle | symbol

%-------------------------------------------------------------------------------------
% Environment shortcuts
%-------------------------------------------------------------------------------------
\def\balign#1\ealign{\begin{align}#1\end{align}}
\def\baligns#1\ealigns{\begin{align*}#1\end{align*}}
\def\balignat#1\ealign{\begin{alignat}#1\end{alignat}}
\def\balignats#1\ealigns{\begin{alignat*}#1\end{alignat*}}
\def\bitemize#1\eitemize{\begin{itemize}#1\end{itemize}}
\def\benumerate#1\eenumerate{\begin{enumerate}#1\end{enumerate}}

% Align environments that use textstyle instead of displaystyle
\newenvironment{talign*}
 {\let\displaystyle\textstyle\csname align*\endcsname}
 {\endalign}
\newenvironment{talign}
 {\let\displaystyle\textstyle\csname align\endcsname}
 {\endalign}

\def\balignst#1\ealignst{\begin{talign*}#1\end{talign*}}
\def\balignt#1\ealignt{\begin{talign}#1\end{talign}}
%---------------------------------------------------

%-------------------------------------------------------------------------------------
%Text with quads around it
%-------------------------------------------------------------------------------------
\newcommand{\qtext}[1]{\quad\text{#1}\quad} 

%-------------------------------------------------------------------------------------
% Redefine left and right to remove initial and trailing space
%-------------------------------------------------------------------------------------
\let\originalleft\left
\let\originalright\right
\renewcommand{\left}{\mathopen{}\mathclose\bgroup\originalleft}
\renewcommand{\right}{\aftergroup\egroup\originalright}

%-------------------------------------------------------------------------------------
% Words with special symbols
%-------------------------------------------------------------------------------------
\def\Gronwall{Gr\"onwall\xspace}
\def\Holder{H\"older\xspace}
\def\Ito{It\^o\xspace}
\def\Nystrom{Nystr\"om\xspace}
\def\Schatten{Sch\"atten\xspace}
\def\Matern{Mat\'ern\xspace}

%-------------------------------------------------------------------------------------
% Smaller citations
%-------------------------------------------------------------------------------------
\def\tinycitep*#1{{\tiny\citep*{#1}}}
\def\tinycitealt*#1{{\tiny\citealt*{#1}}}
\def\tinycite*#1{{\tiny\cite*{#1}}}
\def\smallcitep*#1{{\scriptsize\citep*{#1}}}
\def\smallcitealt*#1{{\scriptsize\citealt*{#1}}}
\def\smallcite*#1{{\scriptsize\cite*{#1}}}

%-------------------------------------------------------------------------------------
% Colors
%-------------------------------------------------------------------------------------
\def\blue#1{\textcolor{blue}{{#1}}}
\def\green#1{\textcolor{green}{{#1}}}
\def\orange#1{\textcolor{orange}{{#1}}}
\def\purple#1{\textcolor{purple}{{#1}}}
\def\red#1{\textcolor{red}{{#1}}}
\def\teal#1{\textcolor{teal}{{#1}}}

%-------------------------------------------------------------------------------------
% Font styles
%-------------------------------------------------------------------------------------
\def\mbi#1{\boldsymbol{#1}} % Bold and italic (math bold italic)
\def\mbf#1{\mathbf{#1}}
\def\mbb#1{\mathbb{#1}}
\def\mc#1{\mathcal{#1}}
\def\mrm#1{\mathrm{#1}}
\def\tbf#1{\textbf{#1}}
\def\tsc#1{\textsc{#1}}
%-------------------------------------------------------------------------------------
% Bold and italic variables
%-------------------------------------------------------------------------------------
\def\mbiA{\mbi{A}}
\def\mbiB{\mbi{B}}
\def\mbiC{\mbi{C}}
\def\mbiDelta{\mbi{\Delta}}
\def\mbif{\mbi{f}}
\def\mbiF{\mbi{F}}
\def\mbih{\mbi{g}}
\def\mbiG{\mbi{G}}
\def\mbih{\mbi{h}}
\def\mbiH{\mbi{H}}
\def\mbiI{\mbi{I}}
\def\mbim{\mbi{m}}
\def\mbiP{\mbi{P}}
\def\mbiQ{\mbi{Q}}
\def\mbiR{\mbi{R}}
\def\mbiv{\mbi{v}}
\def\mbiV{\mbi{V}}
\def\mbiW{\mbi{W}}
\def\mbiX{\mbi{X}}
\def\mbiY{\mbi{Y}}
\def\mbiZ{\mbi{Z}}

%-------------------------------------------------------------------------------------
% Textstyle vs. displaystyle
%-------------------------------------------------------------------------------------
\def\textsum{{\textstyle\sum}} % Sum in textstyle form
\def\textprod{{\textstyle\prod}} % Prod in textstyle form
\def\textbigcap{{\textstyle\bigcap}} % Bigcap in textstyle form
\def\textbigcup{{\textstyle\bigcup}} % Bigcup in textstyle form

%-------------------------------------------------------------------------------------
% Mathematical sets
%-------------------------------------------------------------------------------------
\def\reals{\mathbb{R}} % Real number symbol
\def\integers{\mathbb{Z}} % Integer symbol
\def\rationals{\mathbb{Q}} % Rational numbers
\def\naturals{\mathbb{N}} % Natural numbers
\def\complex{\mathbb{C}} % Complex numbers

%-------------------------------------------------------------------------------------
% Special symbols
%-------------------------------------------------------------------------------------
\def\<{\left\langle} % Angle brackets
\def\>{\right\rangle}

\def\iff{\Leftrightarrow}
%\def\choose#1#2{\left(\begin{array}{c}{#1} \\ {#2}\end{array}\right)}
\def\chooses#1#2{{}_{#1}C_{#2}}
\def\defeq{\triangleq} % defined equal to
\def\bs{\backslash} % backslash
\def\half{\frac{1}{2}}
\def\nhalf{\nicefrac{1}{2}}
\def\textint{{\textstyle\int}} % Sum in textstyle form
\def\texthalf{{\textstyle\frac{1}{2}}}
\newcommand{\textfrac}[2]{{\textstyle\frac{#1}{#2}}}

% Semidefinite orders
\newcommand{\psdle}{\preccurlyeq}
\newcommand{\psdge}{\succcurlyeq}
\newcommand{\psdlt}{\prec}
\newcommand{\psdgt}{\succ}

%-------------------------------------------------------------------------------------
% Vectors and matrices
%-------------------------------------------------------------------------------------
\newcommand{\boldone}{\mbf{1}} % Bold 1
\newcommand{\ident}{\mbf{I}} % Identity matrix
\def\v#1{\mbi{#1}} % Vector notation
\def\norm#1{\left\|{#1}\right\|} % A norm with 1 argument
\newcommand{\onenorm}[1]{\norm{#1}_1} % L1 norm
\newcommand{\twonorm}[1]{\norm{#1}_2} % L2 norm
\newcommand{\infnorm}[1]{\norm{#1}_{\infty}} % Linfty norm
\newcommand{\opnorm}[1]{\norm{#1}_{\text{op}}} % Operator norm
\newcommand{\fronorm}[1]{\norm{#1}_{\text{F}}} % Frobenius norm
\newcommand{\nucnorm}[1]{\norm{#1}_{*}} % Frobenius norm
\def\staticnorm#1{\|{#1}\|} % A static norm that does not resize with input
\newcommand{\statictwonorm}[1]{\staticnorm{#1}_2} % L2 norm
\newcommand{\inner}[1]{{\langle #1 \rangle}} % inner product
\newcommand{\binner}[2]{\left\langle{#1},{#2}\right\rangle} % Inner product with expandable brackets
\def\what#1{\widehat{#1}}

\def\twovec#1#2{\left[\begin{array}{c}{#1} \\ {#2}\end{array}\right]}
\def\threevec#1#2#3{\left[\begin{array}{c}{#1} \\ {#2} \\ {#3} \end{array}\right]}
\def\nvec#1#2#3{\left[\begin{array}{c}{#1} \\ {#2} \\ \vdots \\ {#3}\end{array}\right]} % An n-vector with three arguments

% ------------------------------------------------------------------------
% Eigenvalues
% ------------------------------------------------------------------------
\def\maxeig#1{\lambda_{\mathrm{max}}\left({#1}\right)}
\def\mineig#1{\lambda_{\mathrm{min}}\left({#1}\right)}

%-------------------------------------------------------------------------------------
% Operators
%-------------------------------------------------------------------------------------
\def\Re{\operatorname{Re}} % Real part
\def\indic#1{\mbb{I}\left[{#1}\right]} % Indicator function
\def\logarg#1{\log\left({#1}\right)} % log with argument
\def\polylog{\operatorname{polylog}}
\def\maxarg#1{\max\left({#1}\right)} % max with argument
\def\minarg#1{\min\left({#1}\right)} % min with argument
\def\E{\mbb{E}} % Expectation symbol
\def\Earg#1{\E\left[{#1}\right]}
\def\Esub#1{\E_{#1}}
\def\Esubarg#1#2{\E_{#1}\left[{#2}\right]}
\def\bigO#1{\mathcal{O}\left(#1\right)} % big-oh notation
\def\littleO#1{o(#1)} % big-oh notation
\def\P{\mbb{P}} % Probability symbol
\def\Parg#1{\P\left({#1}\right)}
\def\Psubarg#1#2{\P_{#1}\left[{#2}\right]}
\DeclareMathOperator{\Tr}{Tr} % Trace
\def\T{\top} % transpose
\def\Trarg#1{\Tr\left[{#1}\right]} % Trace with argument
\def\trarg#1{\tr\left[{#1}\right]} % trace with argument
\def\Var{\mrm{Var}} % Variance symbol
\def\Vararg#1{\Var\left[{#1}\right]}
\def\Varsubarg#1#2{\Var_{#1}\left[{#2}\right]}
\def\Cov{\mrm{Cov}} % Covariance symbol
\def\Covarg#1{\Cov\left[{#1}\right]}
\def\Covsubarg#1#2{\Cov_{#1}\left[{#2}\right]}
\def\Corr{\mrm{Corr}} % Covariance symbol
\def\Corrarg#1{\Corr\left[{#1}\right]}
\def\Corrsubarg#1#2{\Corr_{#1}\left[{#2}\right]}
\newcommand{\info}[3][{}]{\mathbb{I}_{#1}\left({#2};{#3}\right)} % Information symbol
%\renewcommand{\exp}[1]{\operatorname{exp}\left(#1\right)} % Exponential
\newcommand{\staticexp}[1]{\operatorname{exp}(#1)} % An exponential with parens that do not resize with input
\newcommand{\loglihood}[0]{\mathcal{L}} % log likelihood

%-------------------------------------------------------------------------------------
% Distributions
%-------------------------------------------------------------------------------------
%\def\normal{{\sf N}}
\newcommand{\Gsn}{\mathcal{N}}
\newcommand{\BeP}{\textnormal{BeP}}
\newcommand{\Ber}{\textnormal{Ber}}
\newcommand{\Bern}{\textnormal{Bern}}
\newcommand{\Bet}{\textnormal{Beta}}
\newcommand{\Beta}{\textnormal{Beta}}
\newcommand{\Bin}{\textnormal{Bin}}
\newcommand{\BP}{\textnormal{BP}}
\newcommand{\Dir}{\textnormal{Dir}}
\newcommand{\DP}{\textnormal{DP}}
\newcommand{\Expo}{\textnormal{Expo}}
\newcommand{\Gam}{\textnormal{Gamma}}
\newcommand{\GEM}{\textnormal{GEM}}
\newcommand{\HypGeo}{\textnormal{HypGeo}}
\newcommand{\Mult}{\textnormal{Mult}}
\newcommand{\NegMult}{\textnormal{NegMult}}
\newcommand{\Poi}{\textnormal{Poi}}
\newcommand{\Pois}{\textnormal{Pois}}
\newcommand{\Unif}{\textnormal{Unif}}


%-------------------------------------------------------------------------------------
% Derivative symbols
%-------------------------------------------------------------------------------------
\newcommand{\grad}{\nabla}
\newcommand{\Hess}{\nabla^2} % Hessian
\newcommand{\lapl}{\triangle} % Laplace operator / Laplacian
\newcommand{\pderiv}[2]{\frac{\partial #1}{\partial #2}} % partial derivative

%-------------------------------------------------------------------------------------
% Probability and statistics macros
%-------------------------------------------------------------------------------------
\newcommand{\eqdist}{\stackrel{d}{=}}
\newcommand{\todist}{\stackrel{d}{\to}}
\newcommand{\eqd}{\stackrel{d}{=}}
\def\KL#1#2{\textnormal{KL}({#1}\Vert{#2})}
\def\independenT#1#2{\mathrel{\rlap{$#1#2$}\mkern4mu{#1#2}}}
%\def\indep{\perp\!\!\!\perp} % conditional independence

%-------------------------------------------------------------------------------------
% Optimization macros
%-------------------------------------------------------------------------------------
\providecommand{\argmax}{\mathop\mathrm{arg max}} % Defining math symbols
\providecommand{\argmin}{\mathop\mathrm{arg min}}
\providecommand{\arccos}{\mathop\mathrm{arccos}}
\providecommand{\dom}{\mathop\mathrm{dom}}
\providecommand{\diag}{\mathop\mathrm{diag}}
\providecommand{\tr}{\mathop\mathrm{tr}}
%\providecommand{\abs}{\mathop\mathrm{abs}}
\providecommand{\card}{\mathop\mathrm{card}}
\providecommand{\sign}{\mathop\mathrm{sign}}
\providecommand{\conv}{\mathop\mathrm{conv}} % Convex hull
\def\rank#1{\mathrm{rank}({#1})}
\def\supp#1{\mathrm{supp}({#1})}

\providecommand{\minimize}{\mathop\mathrm{minimize}}
\providecommand{\maximize}{\mathop\mathrm{maximize}}
\providecommand{\subjectto}{\mathop\mathrm{subject\;to}}

%\renewcommand\eqref[1]{Eq.~(\ref{#1})}

\def\openright#1#2{\left[{#1}, {#2}\right)}

%-------------------------------------------------------------------------------------
% Proof environments
%-------------------------------------------------------------------------------------
\ifdefined\nonewproofenvironments\else
% The Theorems are numbered consecutively
% Lemmas are numbered by section, and observations, claims, facts, and 
% assumptions take their numbering. Propositions and definitions have their
% own numbering by section.
\ifdefined\ispres\else
% These conflict with Beamer definitions in pres mode
\newtheorem{theorem}{Theorem}
\newtheorem{lemma}[theorem]{Lemma}
\newtheorem{corollary}[theorem]{Corollary}
\newtheorem{definition}[theorem]{Definition}
\newtheorem{fact}[theorem]{Fact}
\renewenvironment{proof}{\noindent\textbf{Proof.}\hspace*{.3em}}{\qed\\}
\newenvironment{proof-sketch}{\noindent\textbf{Proof Sketch}
  \hspace*{1em}}{\qed\bigskip\\}
\newenvironment{proof-idea}{\noindent\textbf{Proof Idea}
  \hspace*{1em}}{\qed\bigskip\\}
\newenvironment{proof-of-lemma}[1][{}]{\noindent\textbf{Proof of Lemma {#1}}
  \hspace*{1em}}{\qed\\}
\newenvironment{proof-of-theorem}[1][{}]{\noindent\textbf{Proof of Theorem {#1}}
  \hspace*{1em}}{\qed\\}
\newenvironment{proof-attempt}{\noindent\textbf{Proof Attempt}
  \hspace*{1em}}{\qed\bigskip\\}
\newenvironment{proofof}[1]{\noindent\textbf{Proof of {#1}}
  \hspace*{1em}}{\qed\bigskip\\}
\newenvironment{remark}{\noindent\textbf{Remark.}
  \hspace*{0em}}{\smallskip}%\bigskip}
\newenvironment{remarks}{\noindent\textbf{Remarks}
  \hspace*{1em}}{\smallskip}
\fi
\newtheorem{observation}[theorem]{Observation}
\newtheorem{proposition}[theorem]{Proposition}
\newtheorem{claim}[theorem]{Claim}
\newtheorem{assumption}{Assumption}
%\renewcommand{\theassumption}{\Alph{assumption}} % Set counter for assumptions
                                                 % to be alphabetical
\fi
% Makes equation numbers have (1.1) style
% \numberwithin{equation}{section}
% \numberwithin{equation}{subsection}
\makeatletter
\@addtoreset{equation}{section}
\makeatother
\def\theequation{\thesection.\arabic{equation}}


%-------------------------------------------------------------------------------------
% Equation environments
%-------------------------------------------------------------------------------------
\newcommand{\eq}[1]{\begin{align}#1\end{align}}
\newcommand{\eqn}[1]{\begin{align*}#1\end{align*}}
\renewcommand{\Pr}[1]{\mathbb{P}\left( #1 \right)}
\newcommand{\Ex}[1]{\mathbb{E}\left[#1\right]}
\newcommand{\var}[1]{\text{Var}\left[#1\right]}
\newcommand{\ind}[1]{{\mathbbm{1}}_{\{ #1 \}} }
\newcommand{\abs}[1]{\left|#1\right|}

\newcommand{\paren}[1]{\left(#1\right)}


























